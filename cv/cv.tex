% Jason R. Blevins - Curriculum Vitae
%
% Copyright (C) 2004-2010 Jason R. Blevins
%
% You may use use this document as a template to create your own CV
% and you may redistribute the source code freely.  No attribution is
% required in any resulting documents.  I do ask that you please leave
% this notice and the above URL in the source code if you choose to
% redistribute this file.

\documentclass[10pt,letterpaper]{article}
\synctex=1
\usepackage[colorlinks=false, linktocpage=true]{hyperref}
\usepackage{geometry}

% Fonts
\usepackage[T1]{fontenc}
\usepackage[bitstream-charter]{mathdesign}
\usepackage[scaled=0.85]{beramono}
\usepackage[resetlabels]{multibib}
\usepackage[utf8]{inputenc}
\usepackage{enumitem}
\setlist{noitemsep}

\providecommand{\tightlist}{%
  \setlength{\itemsep}{0pt}\setlength{\parskip}{0pt}}

\newcites{journal}{International Journal Papers}
\newcites{conference}{International Conference Papers}
\newcites{workshop}{International Workshop Papers}
\newcites{technical}{Technical Reports}
\newcites{proceedings}{Edited Volumes}

% Set your name here
\def\name{Massimo Tisi}

% The following metadata will show up in the PDF properties
\hypersetup{
  colorlinks = true,
  urlcolor = black,
  pdfauthor = {\name},
  pdfkeywords = {model-transformation},
  pdftitle = {\name: Curriculum Vitae},
  pdfsubject = {Curriculum Vitae},
  pdfpagemode = UseNone
}

\geometry{
  body={6.5in, 9.0in},
  left=1.0in,
  top=1.0in
}

% Customize page headers
\pagestyle{myheadings}
\markright{\name}
\thispagestyle{empty}

% Custom section fonts
\usepackage{sectsty}
\sectionfont{\rmfamily\mdseries\Large}
\subsectionfont{\rmfamily\mdseries\itshape\large}

% Other possible font commands include:
% \ttfamily for teletype,
% \sffamily for sans serif,
% \bfseries for bold,
% \scshape for small caps,
% \normalsize, \large, \Large, \LARGE sizes.

% Don't indent paragraphs.
\setlength\parindent{0em}

%MAX added to make bibliographies subsections
\makeatletter
\renewenvironment{thebibliography}[1]
     {\subsection*{\refname}%
      \@mkboth{\MakeUppercase\refname}{\MakeUppercase\refname}%
      \list{\@biblabel{\@arabic\c@enumiv}}%
           {\settowidth\labelwidth{\@biblabel{#1}}%
            \leftmargin\labelwidth
            \advance\leftmargin\labelsep
            \@openbib@code
            \usecounter{enumiv}%
            \let\p@enumiv\@empty
            \renewcommand\theenumiv{\@arabic\c@enumiv}}%
      \sloppy
      \clubpenalty4000
      \@clubpenalty \clubpenalty
      \widowpenalty4000%
      \sfcode`\.\@m}
     {\def\@noitemerr
       {\@latex@warning{Empty `thebibliography' environment}}%
      \endlist}
\makeatother
\begin{document}

% Place name at left
{\huge \name}

% Alternatively, print name centered and bold:
%\centerline{\huge \bf \name}

\vspace{0.25in}
\begin{minipage}[t]{0.5\textwidth}
  Associate Professor \\
  Department of automation, \\
    production and computer science (DAPI) \\
  IMT Atlantique \\
  rue Alfred Kastler, BP 20722 \\
  F-44307 Nantes Cedex 3
\end{minipage}
\begin{minipage}[t]{0.5\textwidth}
  Phone: +33 2 5185 8704 \\
  Email: \texttt{massimo.tisi@imt-atlantique.fr} \\
  Homepage: \texttt{https://massimotisi.github.io/}
\end{minipage}

\vfill
\tableofcontents

\vfill
% Footer
\begin{center}
  \begin{small}
    Last updated: \today
  \end{small}
\end{center}

\newpage
\section{Research highlights}

\subsection{Short bio}

Since 2010, I hold an Associate Professor (Maitre Assistant) position at IMT Atlantique (formerly Ecole des Mines de Nantes). 
I obtained my M.Sc. in Computer Science and Engineering at Politecnico di Milano in 2005 (100/100) and my Ph.D. in Information Engineering (mark A) in April 2009, with a thesis on Model Transformations for Artifact Generation in Model-Driven Environments (advisor prof. Piero Fraternali).

Since the beginning of my career, my research focused on the design, implementation and evaluation of novel methods and tools for modeling software systems and reasoning on software models. The scientific work, which resulted in high-profile publications, has always been supported by substantial implementation and demonstration efforts, and by a continuos interaction with industry. While during my doctoral studies I was focused on modeling a precise application domain (data-intensive Web applications), moving to IMT Atlantique I expanded the spectrum of modeled applications. Recently I have started to look outside of purely software systems, by considering hybrid hardware/software models for cyber-physical systems.

During my doctoral studies I have been visiting researcher at the McGill university (Montreal, Canada, 2007), where I came in contact with multi-paradigm heterogeneous modeling. Later I have been several times a visiting researcher at the National Institute of Informatics (NII) in Tokyo, Japan, where I collaborated on advancing the state of the art in bidirectional transformations.

\subsection{Publications and visibility}

I am widely recognized as an expert in model transformation, notably for extending the model-transformation paradigm to higher-order model transformations, and for my contributions to the ATL language since 2010. In particular, my article "On the Use of Higher-Order Model Transformations" has collected 252 citations since 2009 (up to June 2023 - Google Scholar). I have an h-index of 26 (June 2023 - Google Scholar).

\medskip
Awards: 
\begin{itemize}
\item Nominee for Best Paper Award in the European Joint Conferences on Theory and Practice of Software 2017 (ETAPS 2017) for the article: Zheng Cheng, Massimo Tisi, A Deductive Approach for Fault Localization in ATL Model Transformations.
\item Nominee for Best Paper Award in the International Conference for Software Testing 2017 (ICST 2017) for the article: Zheng Cheng, Massimo Tisi, Incremental Deductive Verification for Relational Model Transformations.
\item Best Performance Award for the ATL solution to the Transformation Tool Contest 2017 (TTC 2017), at the Software Technologies: Applications and Foundations Conference 2017 (STAF 2017).
\item Best paper award in the International Conference on Web Engineering 2009 (ICWE 2009) for the article: Fraternali, P., Tisi, M.: A higher-order generative framework for weaving traceability links into a code generator for Web application testing.
\item Best paper award in the International Workshop on Model Driven Web Engineering 2008 (MDWE 2008) for the article: Brambilla, M., Fraternali, P., Tisi, M.: A Metamodel Transformation Framework for the Migration of WebML Models to MDA.
\end{itemize}

\medskip
My top five publications are:
\begin{itemize}
\item Soichiro Hidaka, Frederic Jouault, Massimo Tisi: On Additivity in Transformation Languages. International Conference on Modeling Languages and Systems, MODELS 2017 (to appear).
\item Salvador Martínez, Massimo Tisi, Rémi Douence, Reactive model transformation with ATL, Science of Computer Programming, Volume 136, 2017, Pages 1-16, Elsevier.
\item Soichiro Hidaka, Massimo Tisi, Jordi Cabot, Zhenjiang Hu: Feature-based classification of bidirectional transformation approaches. Software and System Modeling, Volume 15(3), 2016, Pages 907-928, Springer.
\item Piero Fraternali, Massimo Tisi. Using Traceability Links and Higher Order Transformations for Easing Regression Testing of Web Applications. Journal on Web Engineering, Volume 10(1), 2011, Pages 1-20, Rinton.
\item Massimo Tisi, Frédéric Jouault, Piero Fraternali, Stefano Ceri, Jean Bézivin: On the Use of Higher-Order Model Transformations. Proceedings of the 5th European Conference on Model Driven Architecture - Foundations and Applications, 2009, Pages 18 – 33.
\end{itemize}

% \subsection{Funding and Project Management}
% I have worked within several national (Italian and French), and European research projects, as well as with several companies for research exploitation and technology transfer (including Sodifrance, Atos, Thales, and several SMEs). I have been the principal investigator in AtlanMod for four projects, and I have secured 430K Euro in fundings from European and national projects.

% \medskip
% Highlights:
% \begin{itemize}
% \item Principal investigator for France in the MONDO project: Scalable Modeling and Model Management on the Cloud (2013-2016: 7th Framework Programme, STREP 2012). Mention: \textbf{Excellent}.
% \item Principal investigator for France in the AutoMobile project: Automated Mobile App Development (2013-2015: 7th Framework Programme, Research for SMEs). Mention: \textbf{Excellent}.
% \end{itemize}

% \subsection{Research Community Service}
% I am in the program committee of several international conferences and workshops in software engineering, including since 2013 the leading conferences on model transformation (ICMT) and on software language engineering (SLE). I am reviewer for top journals in softw
% are engineering, including Transactions of Software Engineering, Journal of Software: Evolution and Process, Software and System Modeling, Science of Computer Programming. 

% Since 2013 I organize the BigMDE workshop on scalable model-driven engineering, and this year I host the Projects Showcase at the STAF federated conferences 2017.

% I am member of the Monitoring and Evaluation Committee at the Pôle Images \& Réseaux.

% \subsection{Teaching and Academic Supervision}
% Since 2013 I am lecturer of Domain-Specific Languages and Model-Driven Engineering (master level) and of Databases (bachelor level) at IMT Atlantique. I am co-advisor of four PhD students at IMT Atlantique.

\subsection{Main research areas}

\subsection*{Model Transformation Languages}
In the scope of model-driven engineering, model transformation aims to provide a mean to specify the way to produce target models from a number of source models. For this purpose, it enables developers to define the way source model elements must be matched and navigated in order to initialize the target model elements. Formally, a model transformation has to define the way for generating a model Mb, conforming to a metamodel MMb, from a model Ma conforming to a metamodel MMa. In practice, model transformations are widely used in model-driven engineering, e.g. during code generation (their originally intended use case), translation among modeling languages, data translation for interoperability, controlled updates over a model at runtime.

I have investigated different paradigms for model transformation, including the relational paradigm (exemplified by the ATL transformation language), the functional paradigm (exemplified by the OCL language), the data-flow paradigm (exemplified by the fUML language). I have studied correspondences between these paradigms and translations among them. 

I have contributed at clarifying the properties of transformation languages, especially bidirectionality (by feature modeling the design space of bidirectional transformations) and recently additivity, a property relating the syntactic constructs of the transformation specification to their semantic output.  
Finally I have distinguished the important class of higher-order transformations, i.e. transformations that manipulate other transformationss

\subsection*{Model Transformation Verification}

With the increasing use of model-driven engineering in safety-critical domains (e.g., in automotive industry, medical data processing, aviation), it is crucial to develop techniques and tools that prevent incorrect model transformations from generating faulty models. The effects of such faulty models could be unpredictably propagated into subsequent MDE steps, e.g. code generation.

For this reason, there is a great need of mechanisms to ensure quality and the absence of errors in models and model transformations. Verification is one of the effective techniques that are typically used to achieve this.

Verification of models and model transformations refers to the ability of these elements to satisfy one or more correctness properties. These properties express certain characteristics that the element under analysis must feature in order to be considered correct. It is typical for verification tools to use formal methods to determine whether the model or model transformation under analysis satisfies the correctness properties under scrutiny.

I have contributed to formalize the execution semantics of the ATL language by translation to the Boogie intermediate verification language. Based on this semantics I have designed a fault localization method, capable of pinpointing the faulty lines of a transformation that does not respect a contracts.

I have designed a new technique for the decomposition of transformation contracts, to improve the efficiency of the automatic verification process on a SMT solver. I have contributed an incremental verification technique, so that after a change to the system, only the impacted part needs to be reverified. This idea greatly increases the applicability of transformation verification in industrial scenarios. 

\subsection*{Scalability of Model-Driven Engineering}
The increasing adoption of Model-Driven Engineering in industrial contexts highlights scalability as a critical limitation. Indeed, several Model-driven tools show critical efficiency limitations in handling very large models (VLMs), e.g. models made by millions of model elements, not unusual in real-life industrial scenarios. Examples of such models appear both at development time, e.g. while reverse-engineering big systems and at runtime, e.g. coming from a set of sensors, from OpenData repositories or when building applications on social networks. Moreover, the proliferation of models produced as input-outputs of software engineering tasks at development/maintenance time also highlights scalability problems in the management of the model artifacts.

In order to tackle the scalability problem I research solutions on three main axes:

\begin{itemize}

\item Efficient Transformation. Performing only the strictly required computation on models improves scalability, as only parts of VLM need to be loaded and manipulated. I provided the following support for the efficient execution of model-to-model transformations:
1) An engine for the incremental execution of model-to-model transformations, performing only the necessary re-computations; 2) A lazy execution semantics for model transformations, delaying the computation to when it is strictly needed; 3) An engine for the event-driven reactive execution of model transformations, where the engine performs only the computation needed to react to model updates and requests from the user application.

\item Parallel Transformation. The previous techniques increase efficiency by avoiding unnecessary computation. However, this is not always possible, as several transformations perform global algorithms on the whole model. In this scenario, a solution for the scalability problem would be the parallelization of the transformation with the aim of  decomposing it in smaller independent problems, more easily manageable with current tools. In this context, I conceived 1) A shared-memory implicit parallel execution engine for the ATL transformation language and 2) A distributed framework for model transformation on the Map-Reduce programming model. In particular automatic model distribution is performed within the cluster, by a streaming partitioning algorithm over the input model.

\item Efficient Model Persistence. Very large models need high-performance mechanism for their storage and access. Traditional approaches are file-based, especially XML-based, or rely on relational databases. I investigated innovative mechanisms for model storage, by relying on map-databases, graph-databases and distributed hash-tables. These approches have recently converged in the NeoEMF model persistence tool.

\end{itemize}

\newpage
\section{Research projects}
{
  Between 2019 and 2023 I coordinate the
\href{https://www.lowcomote.eu/}{Lowcomote} project, a Marie Curie
European Training Network (ETN) on Low-Code Engineering Platforms.

I am (or have been) principal investigator for the
\href{http://web.emn.fr/x-info/atlanmod/index.php?title=Main_Page}{AtlanMod}
or NaoMod team for the following funded collaborative projects (besides
participating in
\href{http://web.emn.fr/x-info/atlanmod/index.php?title=Projects}{several
others}): * \href{https://www.aidoart.eu/}{AIDOaRT}: H2020-ECSEL project
on AI-augmented automation supporting modeling, coding, testing,
monitoring, and continuous development in Cyber-Physical Systems
(Co-investigator, 2021-2024) *
\href{https://www.lowcomote.eu/}{Lowcomote}: Marie Curie European
Training Network (ETN) on Low-Code Engineering Platforms (Coordinator,
2019-2022) * \href{http://massimotisi.github.io}{CYCLOPS}: ANR MRSEI
project on Domain-Specific Model-Centric Engineering of Smart
Cyber-Physical Systems of Systems (Coordinator, 2019-2021) *
\href{https://megamart2-ecsel.eu/}{MegaM@RT2}: ECSEL project on a
scalable model-based framework for continuous development and runtime
validation of complex systems (Co-investigator, 2017-2020) *
\href{http://www.mondo-project.org/}{MONDO}: EC FP7 STREP project on
scalable model-driven engineering (Principal Investigator, 2013-2016) *
\href{http://automobile.webratio.com/}{AutoMobile}: EC FP7 Research for
SMEs project on Automated Mobile App Development (Principal
Investigator, 2013-2015) *
\href{http://www.irccyn.ec-nantes.fr/fr/projets-ivc/projet-streammaster-ivc}{StreamMaster}:
Pole Images et Reseaux PME 2011 project on Smart Management of Document
Streams (Principal Investigator, 2012-2014) *
\href{https://itea3.org/project/opees.html}{OPEES}: ITEA 2 Call 3
project on Open Platform for the Engineering of Embedded Systems
(Principal Investigator, 2009-2012)

When I was in Politecnico di Milano, I participated in several research
projects, including:

\begin{itemize}
\tightlist
\item
  \href{http://www.fondazionepolitecnico.it/it/cosa-facciamo/progetti-di-innovazione/item/energy-ch-it-distretto-per-le-tecnologie-e-i-materiali-per-l-efficienza-energetica-dell-insubria}{Energy
  CH-IT}
\item
  \href{https://cordis.europa.eu/project/rcn/88591/factsheet/en}{SECO:
  Search Computing}
\item
  \href{http://www.fondazionepolitecnico.it/it/cosa-facciamo/progetti-di-innovazione/item/bisf-business-e-innovazione-senza-frontiere}{BISF:
  Business e Innovazione senza Frontiere}
\item
  \href{http://autonomamente.como.polimi.it/index85f3.html?option=com_content\&task=view\&id=15\&Itemid=16}{AutonomaMente}
\end{itemize}

}

\newpage  
\section{Supervision}
{
  \let\section\paragraph
  Since 2022 I am member of the council of the
\href{https://ed-spin.doctorat-bretagne.fr/}{SPIN Doctoral School} and
co-responsible for Doctoral Affairs at \href{http://www.ls2n.fr/}{LS2N
(UMR CNRS 6004)}.

I have (co-)supervised the following PhD students, post-doctoral
researchers and research engineers:

\begin{itemize}
\item
  Matthew Coyle (PhD Student - ongoing)
\item
  Josselin Enet (PhD Student - ongoing)
\item
  James Pontes Miranda (PhD Student - ongoing)
\item
  Zahra Rajaei (PhD Student - ongoing)
\item
  Jolan Philippe (PhD Student)
\item
  Joachim Hotonnier (PhD Student)
\item
  Imad Berruyne (PhD Student, excellent mention)
\item
  Thibault Béziers la Fosse (PhD Student)
\item
  Zheng Cheng (Post-doc researcher)
\item
  Gwendal Daniel (PhD Student, Research Engineer)
\item
  Amine Benelallam (PhD Student)
\item
  Salvador Martinez Pérez (Post-doc researcher)
\item
  Valerio Cosentino (Post-doc researcher)
\item
  Zied Saidi (Research Engineer)
\item
  Abel Gómez Llana (Post-doc researcher)
\item
  Hassene Choura (Research Engineer)
\end{itemize}

I have participated to the supervision committee of the following phd
students:

\begin{itemize}
\tightlist
\item
  Chahrazed Boudjemila (IMT Atlantique)
\item
  Maxime Mere (INSA Rennes)
\item
  Vianney Sicard (INRAE)
\item
  Theo Le Calvar (ESEO Angers)
\end{itemize}

I have participated to the thesis review or defense jury of the
following phd students:

\begin{itemize}
\tightlist
\item
  Vianney Sicard (INRAE)
\item
  Antonio Garmendía (Universidad Autonoma de Madrid)
\item
  Loli Burgueño (Universidad de Málaga)
\item
  Elena Planas Hortal (Universitat Oberta de Catalunya)
\item
  Romeo Marinelli (Università dell'Aquila)
\item
  Medhi Iraqi (Ecole Nationale Supérieure d'Arts et Métiers - CER
  d'Aix-en-Provence)
\end{itemize}

}
 
\newpage  
\section{Conference organization and referee service}
{
  \let\section\paragraph
  I have served as co-organizer and reviewer for the following venues.

\hypertarget{section}{%
\section{2023}\label{section}}

\begin{itemize}
\tightlist
\item
  \href{http://www.images-et-reseaux.com/en}{Monitoring and Evaluation
  Committee at the Pôle Images \& Réseaux} (Member)
\item
  \href{https://modelsconference.org/}{International Conference on Model
  Driven Engineering Languages and Systems (MODELS)} (Workshops PC)
\item
  \href{http://www.sosym.org/}{Software and System Modeling (SoSyM)}
  (Journal reviewer)
\end{itemize}

\hypertarget{section-1}{%
\section{2022}\label{section-1}}

\begin{itemize}
\tightlist
\item
  \href{http://www.images-et-reseaux.com/en}{Monitoring and Evaluation
  Committee at the Pôle Images \& Réseaux} (Member)
\item
  \href{https://staf2022.univ-nantes.io/}{Software Technologies:
  Applications and Foundations (STAF)} (General Chair)
\item
  \href{https://lowcode-workshop.github.io/}{Lowcode Workshop (Lowcode)}
  (Organizer)
\item
  \href{https://conf.researchr.org/home/icse-2022/gas-2022}{International
  Workshop on Games and Software Engineering (GAS)} (PC)
\item
  \href{https://mleworkshop.github.io/editions/mle2022/}{International
  Workshop on Modeling Language Engineering (MLE)} (PC)
\item
  \href{http://www.sosym.org/}{Software and System Modeling (SoSyM)}
  (Journal reviewer)
\end{itemize}

\hypertarget{section-2}{%
\section{2021}\label{section-2}}

\begin{itemize}
\tightlist
\item
  \href{http://www.images-et-reseaux.com/en}{Monitoring and Evaluation
  Committee at the Pôle Images \& Réseaux} (Member)
\item
  \href{https://modelsconference.org/}{International Conference on Model
  Driven Engineering Languages and Systems (MODELS)} (PC, Artifact
  Evaluation Chair)
\item
  \href{https://lowcode-workshop.github.io/}{Lowcode Workshop (Lowcode)}
  (Organizer)
\item
  \href{https://www.apms-conference.org/past-conferences/apms-2021/}{Advances
  in Production Management Systems (APMS)}(Special Session Organizer)
\item
  \href{https://staf2019.win.tue.nl/events/ecmfa20/}{European Conference
  on Modelling Foundations and Applications (ECMFA)} (PC)
\item
  \href{https://gemoc.org/events/moddit2021.html}{International Workshop
  on Model-Driven Engineering of Digital Twins (MODDIT)} (PC)
\item
  \href{https://www.computer.org/web/tse}{IEEE Transactions of Software
  Engineering (TSE)}(Journal Reviewer)
\item
  \href{http://www.sosym.org/}{Software and System Modeling (SoSyM)}
  (Journal reviewer)
\end{itemize}

\hypertarget{section-3}{%
\section{2020}\label{section-3}}

\begin{itemize}
\tightlist
\item
  \href{http://www.images-et-reseaux.com/en}{Monitoring and Evaluation
  Committee at the Pôle Images \& Réseaux} (Member)
\item
  \href{https://lowcode-workshop.github.io/}{Lowcode Workshop (Lowcode)}
  (Organizer)
\item
  \href{https://modelsconference.org/}{International Conference on Model
  Driven Engineering Languages and Systems (MODELS)} (PC)
\item
  \href{https://staf2019.win.tue.nl/events/ecmfa20/}{European Conference
  on Modelling Foundations and Applications (ECMFA)} (PC)
\item
  \href{http://www.biscuits.work/fourth-workshop/}{Workshop on Software
  Foundations for Data Interoperability @VLDB (SFDI)} (PC)
\item
  \href{http://www.sosym.org/}{Software and System Modeling (SoSyM)}
  (Journal reviewer)
\item
  \href{http://interfacce.mifav.uniroma2.it/inevent/events/idea2010/?s=9}{Interaction
  Design and Architecture(s) Journal (IxD\&A)} (Journal reviewer)
\end{itemize}

\hypertarget{section-4}{%
\section{2019}\label{section-4}}

\begin{itemize}
\tightlist
\item
  \href{https://staf2019.win.tue.nl/}{Software Technologies:
  Applications and Foundations (STAF)} (Workshop Chair)
\item
  \href{http://www.images-et-reseaux.com/en}{Monitoring and Evaluation
  Committee at the Pôle Images \& Réseaux} (Member)
\item
  \href{https://modelsconf19.org/}{International Conference on Model
  Driven Engineering Languages and Systems (MODELS)} (PC)
\item
  \href{https://staf2019.win.tue.nl/events/ecmfa19/}{European Conference
  on Modelling Foundations and Applications (ECMFA)} (PC)
\item
  \href{http://www.model-transformation.org/}{International Conference
  on Model Transformation (ICMT)} (PC)
\item
  \href{https://staf2019.win.tue.nl/events/staf-jrc19/}{STAF Junior
  Researcher Community Event (JRC)} (PC)
\item
  \href{https://staf2019.win.tue.nl/events/staf-rps19/}{STAF Research
  Project Showcase Workshop} (PC)
\item
  \href{http://www.sosym.org/}{Software and System Modeling (SoSyM)}
  (Journal reviewer)
\item
  \href{http://www.journals.elsevier.com/journal-of-systems-and-software}{Journal
  of Systems and Software (JSS)} (Journal reviewer)
\end{itemize}

\hypertarget{section-5}{%
\section{2018}\label{section-5}}

\begin{itemize}
\tightlist
\item
  \href{http://www.images-et-reseaux.com/en}{Monitoring and Evaluation
  Committee at the Pôle Images \& Réseaux} (Member)
\item
  \href{http://measure.softeam-rd.eu/events-workshops/itea3measureindustrialworkshopeventinnanteson15thjune2018}{MEASURE
  Industrial Workshop} (Keynote speaker)
\item
  \href{http://eventmall.info/ecmfa2018/}{European Conference on
  Modelling Foundations and Applications (ECMFA)} (PC, Session chair)
\item
  \href{http://icwe2018.webengineering.org/}{International Conference on
  Web Engineering (ICWE)} (PC)
\item
  \href{https://icmt2018.github.io/}{International Conference on Model
  Transformation (ICMT)} (PC, Session chair)
\item
  \href{http://www.transformation-tool-contest.eu/}{Transformation Tool
  Contest (TTC)} (PC)
\item
  \href{http://icwe2018.webengineering.org/}{Short Papers Track at the
  International Conference on Web Engineering (ICWE)} (PC)
\item
  \href{https://megamart2-ecsel.eu/mde-derun-2018/}{Workshop on
  Model-Driven Engineering for Design-Runtime Interaction in Complex
  Systems (MDE@DeRun)} (PC)
\item
  \href{http://www.models-and-evolution.com/2018/}{Workshop on Models
  and Evolution (ME)} (PC)
\item
  \href{http://www.modre2018.ece.mcgill.ca/}{Workshop on Model-Driven
  Requirements Engineering (MoDRE)} (PC)
\item
  \href{https://oclworkshop.github.io/2018/}{Workshop on the Object
  Constraint Language (OCL)} (PC)
\item
  \href{https://msdl.uantwerpen.be/conferences/MDEbug/2018/}{Workshop on
  Debugging in Model-Driven Engineering (MDEbug)} (PC)
\item
  \href{http://www.modelexecution.org/?page_id=2173}{Workshop on
  Executable Modeling (EXE)} (PC)
\item
  \href{http://www.journals.elsevier.com/science-of-computer-programming/}{Science
  of Computer Programming (SciCO)} (Journal reviewer)
\item
  \href{http://www.sosym.org/}{Software and System Modeling (SoSyM)}
  (Journal reviewer)
\item
  \href{http://www.journals.elsevier.com/journal-of-systems-and-software}{Journal
  of Systems and Software (JSS)} (Journal reviewer)
\end{itemize}

\hypertarget{section-6}{%
\section{2017}\label{section-6}}

\begin{itemize}
\tightlist
\item
  \href{http://www.images-et-reseaux.com/en}{Monitoring and Evaluation
  Committee at the Pôle Images \& Réseaux} (Member)
\item
  \href{http://www.informatik.uni-marburg.de/staf2017/index.php/projects-showcases/}{Project
  Showcase at Software Technologies: Applications and Foundations
  (STAF)} (Co-organizer)
\item
  \href{http://www.big-mde.eu/}{Workshop on Scalable Model-Driven
  Engineering (BigMDE)} (Co-organizer)
\item
  \href{http://www.model-transformation.org/}{International Conference
  on Model Transformation (ICMT)} (PC)
\item
  \href{http://icwe2017.webengineering.org/}{International Conference on
  Web Engineering (ICWE)} (PC)
\item
  \href{http://www.cs.colostate.edu/~ghosh/models17_td/home_models17_td.html}{Tools
  and Demos at MoDELS Conference (MoDELS)} (PC)
\item
  \href{http://icwe2017.webengineering.org/}{Short Papers Track at the
  International Conference on Web Engineering (ICWE)} (PC)
\item
  \href{https://ciel2016.sciencesconf.org/}{Conférence en Ingénierie du
  Logiciel (CIEL)} (PC)
\item
  \href{http://www.transformation-tool-contest.eu/}{Transformation Tool
  Contest (TTC)} (PC)
\item
  \href{http://www.modre2017.ece.mcgill.ca/}{Workshop on Model-Driven
  Requirements Engineering (MoDRE)} (PC)
\item
  \href{http://oclworkshop.github.io/2017/}{Workshop on the Object
  Constraint Language (OCL)} (PC)
\item
  \href{https://msdl.uantwerpen.be/conferences/MDEbug/}{Workshop on
  Debugging in Model-Driven Engineering (MDEbug)} (PC)
\item
  \href{http://www.modelexecution.org/?page_id=1820}{Workshop on
  Executable Modeling (EXE)} (PC)
\item
  \href{http://www.journals.elsevier.com/science-of-computer-programming/}{Science
  of Computer Programming (SciCO)} (Journal reviewer)
\item
  \href{https://www.journals.elsevier.com/science-of-computer-programming/call-for-software/a-new-software-track-on-original-software-publications-scico/}{Original
  Software Publications in Science of Computer Programming} (Journal
  reviewer)
\item
  \href{http://www.sosym.org/}{Software and System Modeling (SoSyM)}
  (Journal reviewer)
\item
  \href{http://www.journals.elsevier.com/journal-of-systems-and-software}{Journal
  of Systems and Software (JSS)} (Journal reviewer)
\end{itemize}

\hypertarget{section-7}{%
\section{2016}\label{section-7}}

\begin{itemize}
\tightlist
\item
  \href{http://www.big-mde.eu/}{Workshop on Scalable Model-Driven
  Engineering (BigMDE)} (Co-organizer)
\item
  \href{http://is.ieis.tue.nl/research/ICMT16/}{International Conference
  on Model Transformation (ICMT)} (PC)
\item
  \href{http://2016.splashcon.org/track/itsle2016}{Industry Track for
  Software Language Engineering (ITSLE)} (PC)
\item
  \href{http://www.transformation-tool-contest.eu/}{Transformation Tool
  Contest (TTC)} (PC)
\item
  \href{http://models2016.irisa.fr/tool-demonstrations/}{Tool
  Demonstrations at the MoDELS Conference} (PC)
\item
  \href{http://www.modre2016.ece.mcgill.ca/}{Workshop on Model-Driven
  Requirements Engineering (MoDRE)} (PC)
\item
  \href{http://oclworkshop.github.io/2016/news.html}{Workshop on the
  Object Constraint Language (OCL)} (PC)
\item
  \href{http://www.modelexecution.org/?page_id=1743}{Workshop on
  Executable Modeling (EXE)} (PC)
\item
  \href{http://www.models-and-evolution.com/2016/}{Workshop on Models
  and Evolution (ME)} (PC)
\item
  \href{https://www.computer.org/web/tse;jsessionid=6afd856a99689b17c0c58edc329c}{IEEE
  Transactions on Software Engineering (TSE)} (Journal reviewer)
\item
  \href{http://www.sosym.org/}{Software and System Modeling (SoSyM)}
  (Journal reviewer)
\item
  \href{http://www.journals.elsevier.com/science-of-computer-programming/}{Science
  of Computer Programming (SciCO)} (Journal reviewer)
\end{itemize}

\hypertarget{section-8}{%
\section{2015}\label{section-8}}

\begin{itemize}
\tightlist
\item
  \href{https://big-mde.github.io/2015.html}{Workshop on Scalable
  Model-Driven Engineering (BigMDE)} (Co-organizer)
\item
  \href{http://www.di.univaq.it/diruscio/sites/ICMT2015/}{International
  Conference on Model Transformation (ICMT)} (PC)
\item
  \href{http://www.sofsem.cz/sofsem15/}{International Conf. on Current
  Trends in Theory and Practice of Computer Science (SOFSEM)} (PC)
\item
  \href{http://www.modre2015.ece.mcgill.ca/}{Workshop on Model-Driven
  Requirements Engineering (MoDRE)} (PC)
\item
  \href{http://www.modelexecution.org/?page_id=1619}{Workshop on
  Executable Modeling (EXE)} (PC)
\item
  \href{https://www.computer.org/web/tse;jsessionid=6afd856a99689b17c0c58edc329c}{IEEE
  Transactions on Software Engineering (TSE)} (Journal reviewer)
\item
  \href{http://onlinelibrary.wiley.com/journal/10.1002/(ISSN)2047-7481}{Journal
  of Software: Evolution and Process} (Journal reviewer)
\item
  \href{http://www.sosym.org/}{Software and System Modeling (SoSyM)}
  (Journal reviewer)
\end{itemize}

\hypertarget{section-9}{%
\section{2014}\label{section-9}}

\begin{itemize}
\tightlist
\item
  \href{https://big-mde.github.io/2014.html}{Workshop on Scalable
  Model-Driven Engineering (BigMDE)} (Co-organizer)
\item
  \href{http://www.sleconf.org/2014/Panel.html}{Education of Language
  Engineers at SLE} (Panelist)
\item
  \href{https://rth3.wp.mines-telecom.fr/journees-rt3/}{Model-Driven
  Engineering at Journées RT3} (Panelist)
\item
  \href{http://www.di.univaq.it/ICMT2014/}{International Conference on
  Model Transformation (ICMT)} (PC)
\item
  \href{http://sofsem14.ics.upjs.sk/}{International Conf. on Current
  Trends in Theory and Practice of Computer Science (SOFSEM)} (PC)
\item
  \href{http://www.sleconf.org/2014/ITSLE.html}{Industry Track for
  Software Language Engineering (ITSLE)} (PC)
\item
  \href{http://www.transformation-tool-contest.eu/2014/}{Transformation
  Tool Contest (TTC)} (PC)
\item
  \href{http://www.journals.elsevier.com/the-journal-of-logic-and-algebraic-programming}{Journal
  of Logic and Algebraic Programming (JLAP)} (Journal reviewer)
\item
  \href{http://www.journals.elsevier.com/neurocomputing}{NeuroComputing}
  (Journal reviewer)
\item
  \href{http://www.rintonpress.com/journals/jwe/}{Journal of Web
  Engineering (JWE)} (Journal reviewer)
\item
  \href{http://ecmfa2014.lcc.uma.es/\#}{European Conference on Modelling
  Foundations and Applications (ECMFA)} (Reviewer)
\end{itemize}

\hypertarget{and-before}{%
\section{2013 and before}\label{and-before}}

\begin{itemize}
\tightlist
\item
  \href{https://big-mde.github.io/2013.html}{Workshop on Scalable
  Model-Driven Engineering (BigMDE) 2013} (Co-organizer)
\item
  \href{http://web.emn.fr/x-info/atlanmod/index.php?title=MtATL2010}{Workshop
  on Model transformations with ATL (MtATL) 2010} (Co-organizer)
\item
  \href{http://web.emn.fr/x-info/atlanmod/index.php?title=MtATL2009}{Workshop
  on Model transformations with ATL (MtATL) 2009} (Co-organizer)
\item
  \href{http://www.model-transformation.org/ICMT2013/}{International
  Conference on Model Transformation (ICMT) 2013} (PC)
\item
  \href{http://icwe2010.webengineering.org/Calls/demos.aspx}{Demos and
  Posters Track at the International Conference on Web Engineering 2010
  (ICWE) 2010} (PC)
\item
  \href{https://www.computer.org/web/tse;jsessionid=6afd856a99689b17c0c58edc329c}{IEEE
  Transactions on Software Engineering (TSE)} (Journal reviewer)
\item
  \href{http://www.journals.elsevier.com/science-of-computer-programming/}{Science
  of Computer Programming (SciCO)} (Journal reviewer)
\item
  \href{http://www.journals.elsevier.com/the-journal-of-logic-and-algebraic-programming}{Journal
  of Logic and Algebraic Programming (JLAP)} (Journal reviewer)
\item
  \href{https://www.computer.org/software-magazine/}{IEEE Software}
  (Journal reviewer)
\item
  \href{http://www.journals.elsevier.com/journal-of-systems-and-software}{Journal
  of Systems and Software (JSS)} (Journal reviewer)
\item
  \href{http://www.sosym.org/}{Software and System Modeling (SoSyM)}
  (Journal reviewer)
\item
  \href{http://onlinelibrary.wiley.com/journal/10.1002/(ISSN)1097-024X}{Software:
  Practice and Experience} (Journal reviewer)
\item
  \href{http://www.rintonpress.com/journals/jwe/}{Journal of Web
  Engineering (JWE)} (Journal reviewer)
\item
  \href{http://www.journals.elsevier.com/information-processing-letters/}{Information
  Processing Letters (IPL)} (Journal reviewer)
\item
  \href{http://models2010.ifi.uio.no/}{International Conference on Model
  Driven Engineering Languages and Systems (MoDELS) 2010} (Reviewer)
\item
  \href{http://www.model-transformation.org/ICMT2010/}{International
  Conference on Model Transformation (ICMT) 2010} (Reviewer)
\item
  \href{http://icwe2010.webengineering.org/}{International Conference on
  Web Engineering (ICWE) 2010} (Reviewer)
\item
  \href{http://i.cs.hku.hk/icde2009/}{International Conference on Data
  Engineering (ICDE) 2009} (Reviewer)
\item
  \href{http://www.model-transformation.org/ICMT2009/}{International
  Conference on Model Transformation (ICMT) 2009} (Reviewer)
\item
  \href{http://www.servicescongress.org/2009/1/}{International
  Conference on Web Services (ICWS) 2009} (Reviewer)
\item
  \href{http://wwwconference.org/www2008/}{International World Wide Web
  Conference (WWW) 2008} (Reviewer)
\item
  \href{http://icwe2008.webengineering.org/}{International Conference on
  Web Engineering (ICWE) 2008} (Reviewer)
\item
  \href{http://www.servicescongress.org/2009/1/}{International
  Conference on Web Services (ICWS) 2008} (Reviewer)
\item
  \href{http://icwe.como.polimi.it/}{International Conference on Web
  Engineering (ICWE) 2008} (Reviewer)
\item
  \href{http://wise2007.loria.fr/pmwiki/pmwiki.php}{International
  Conference on Web Information Systems Engineering (WISE) 2007}
  (Reviewer)
\end{itemize}

}

\newpage
\section{Teaching activities}
{
  \let\section\paragraph
  I have been responsible for the following courses and student projects
at \href{https://www.imt-atlantique.fr/}{IMT Atlantique}, formerly at
Ecole de Mines de Nantes (a student account is required to access most
of the links).

\hypertarget{section}{%
\section{2022-2023}\label{section}}

\begin{itemize}
\tightlist
\item
  \href{https://moodle.imt-atlantique.fr/course/view.php?id=1481}{Software
  Engineering and Modeling} (FIT-A2)
\item
  \href{https://moodle.imt-atlantique.fr/course/view.php?id=1486}{Domain-Specific
  Languages and Model-Driven Engineering} (FIL-A3)
\item
  \href{https://formations.imt-atlantique.fr/bd_ihm}{Databases}
  (FING-A1)
\item
  \href{https://moodle.imt-atlantique.fr/course/view.php?id=299}{Sensibilization
  to Research: MDE} (FIL-A3)
\item
  \href{https://moodle.imt-atlantique.fr/course/view.php?id=16}{Collaborative
  Development} (FING-A1)
\item
  \href{https://moodle.imt-atlantique.fr/course/view.php?id=314}{PFE
  Project} (FING-A3)
\item
  \href{https://moodle.imt-atlantique.fr/course/view.php?id=314}{PFE
  Project} (FIL-A3)
\end{itemize}

\hypertarget{section-1}{%
\section{2021-2022}\label{section-1}}

\begin{itemize}
\tightlist
\item
  \href{https://moodle.imt-atlantique.fr/course/view.php?id=1481}{Software
  Engineering and Modeling} (FIT-A2)
\item
  \href{https://moodle.imt-atlantique.fr/course/view.php?id=1486}{Domain-Specific
  Languages and Model-Driven Engineering} (FIL-A3)
\item
  \href{https://formations.imt-atlantique.fr/bd_ihm}{Databases}
  (FING-A1)
\item
  \href{https://moodle.imt-atlantique.fr/course/view.php?id=299}{Sensibilization
  to Research: MDE} (FIL-A3)
\item
  \href{https://moodle.imt-atlantique.fr/course/view.php?id=16}{Collaborative
  Development} (FING-A1)
\item
  \href{https://moodle.imt-atlantique.fr/course/view.php?id=314}{PFE
  Project} (FING-A3)
\item
  \href{https://moodle.imt-atlantique.fr/course/view.php?id=314}{PFE
  Project} (FIL-A3)
\end{itemize}

\hypertarget{section-2}{%
\section{2020-2021}\label{section-2}}

\begin{itemize}
\tightlist
\item
  \href{https://moodle.imt-atlantique.fr/course/view.php?id=313}{Domain-Specific
  Languages and Model-Driven Engineering} (FING-A3-GSI, FIL-A3)
\item
  \href{https://formations.imt-atlantique.fr/bd_ihm}{Databases}
  (FING-A1)
\item
  \href{https://moodle.imt-atlantique.fr/course/view.php?id=299}{Sensibilization
  to Research: MDE} (FIL-A3)
\item
  \href{https://moodle.imt-atlantique.fr/course/view.php?id=16}{Collaborative
  Development} (FING-A1)
\item
  \href{https://moodle.imt-atlantique.fr/course/view.php?id=314}{PFE
  Project} (FING-A3)
\item
  \href{https://moodle.imt-atlantique.fr/course/view.php?id=314}{PFE
  Project} (FIL-A3)
\end{itemize}

\hypertarget{section-3}{%
\section{2019-2020}\label{section-3}}

\begin{itemize}
\tightlist
\item
  \href{https://moodle.imt-atlantique.fr/course/view.php?id=313}{Domain-Specific
  Languages} (FING-A3-GSI, FIL-A3)
\item
  \href{https://moodle.imt-atlantique.fr/course/view.php?id=668}{Model-Driven
  Engineering} (FIL-A3)
\item
  \href{https://formations.imt-atlantique.fr/bd_ihm}{Databases}
  (FING-A1)
\item
  \href{https://moodle.imt-atlantique.fr/course/view.php?id=299}{Sensibilization
  to Research: MDE} (FIL-A3)
\item
  \href{https://moodle.imt-atlantique.fr/course/view.php?id=16}{Collaborative
  Development} (FING-A1)
\item
  \href{https://moodle.imt-atlantique.fr/course/view.php?id=314}{PFE
  Project} (FING-A3)
\item
  \href{https://moodle.imt-atlantique.fr/course/view.php?id=314}{PFE
  Project} (FIL-A3)
\end{itemize}

\hypertarget{section-4}{%
\section{2018-2019}\label{section-4}}

\begin{itemize}
\tightlist
\item
  \href{https://campusneo.mines-nantes.fr/campus/course/view.php?id=1767}{Domain-Specific
  Languages} (FING-A3-GSI, FIL-A3)
\item
  \href{https://campusneo.mines-nantes.fr/campus/course/view.php?id=1777}{Model-Driven
  Engineering} (FIL-A3)
\item
  \href{https://formations.imt-atlantique.fr/bd_ihm}{Databases}
  (FING-A1)
\item
  \href{https://campusneo.mines-nantes.fr/campus/course/view.php?id=1532}{Sensibilization
  to Research: MDE} (FIL-A3)
\item
  \href{https://campusneo.mines-nantes.fr/campus/course/view.php?id=621}{PRIME
  Project} (FING-A1)
\item
  \href{https://campusneo.mines-nantes.fr/campus/course/view.php?id=1540}{PFE
  Project} (FING-A3)
\item
  \href{https://campusneo.mines-nantes.fr/campus/course/view.php?id=1344}{PFE
  Project} (FIL-A3)
\end{itemize}

\hypertarget{section-5}{%
\section{2017-2018}\label{section-5}}

\begin{itemize}
\tightlist
\item
  \href{https://campusneo.mines-nantes.fr/campus/course/view.php?id=1687}{Domain-Specific
  Languages} (FING-A3-GSI, FIL-A3)
\item
  \href{https://campusneo.mines-nantes.fr/campus/course/view.php?id=1688}{Model-Driven
  Engineering} (FIL-A3)
\item
  \href{https://campusneo.mines-nantes.fr/campus/course/view.php?id=1679}{Databases}
  (FING-A1)
\item
  \href{https://campusneo.mines-nantes.fr/campus/course/view.php?id=1532}{Sensibilization
  to Research: MDE} (FIL-A3)
\item
  \href{https://campusneo.mines-nantes.fr/campus/course/view.php?id=621}{PRIME
  Project} (FING-A1)
\item
  \href{https://campusneo.mines-nantes.fr/campus/course/view.php?id=115}{PIST
  Project} (FING-A2)
\item
  \href{https://campusneo.mines-nantes.fr/campus/course/view.php?id=1540}{PFE
  Project} (FING-A3)
\item
  \href{https://campusneo.mines-nantes.fr/campus/course/view.php?id=1344}{PFE
  Project} (FIL-A3)
\end{itemize}

\hypertarget{section-6}{%
\section{2016-2017}\label{section-6}}

\begin{itemize}
\tightlist
\item
  \href{https://campusneo.mines-nantes.fr/campus/course/view.php?id=1609}{Domain-Specific
  Languages} (FING-A3-GSI, FIL-A3)
\item
  \href{https://campusneo.mines-nantes.fr/campus/course/view.php?id=1628}{Model-Driven
  Engineering} (FIL-A3)
\item
  \href{https://campusneo.mines-nantes.fr/campus/course/view.php?id=1598}{Databases}
  (FING-A1)
\item
  \href{https://campusneo.mines-nantes.fr/campus/course/view.php?id=1532}{Sensibilization
  to Research: MDE} (FIL-A3)
\item
  \href{https://campusneo.mines-nantes.fr/campus/course/view.php?id=621}{PRIME
  Project} (FING-A1)
\item
  \href{https://campusneo.mines-nantes.fr/campus/course/view.php?id=115}{PIST
  Project} (FING-A2)
\item
  \href{https://campusneo.mines-nantes.fr/campus/course/view.php?id=1540}{PFE
  Project} (FING-A3)
\item
  \href{https://campusneo.mines-nantes.fr/campus/course/view.php?id=1344}{PFE
  Project} (FIL-A3)
\end{itemize}

\hypertarget{section-7}{%
\section{2015-2016}\label{section-7}}

\begin{itemize}
\tightlist
\item
  \href{https://campusneo.mines-nantes.fr/campus/course/view.php?id=1571}{Domain-Specific
  Languages} (FING-A3-GSI, FIL-A3)
\item
  \href{https://campusneo.mines-nantes.fr/campus/course/view.php?id=1558}{Model-Driven
  Engineering} (FIL-A3)
\item
  \href{https://campusneo.mines-nantes.fr/campus/course/view.php?id=1545}{Databases}
  (FING-A2)
\item
  \href{https://campusneo.mines-nantes.fr/campus/course/view.php?id=1546}{Databases}
  (FING-A1)
\item
  \href{https://campusneo.mines-nantes.fr/campus/course/view.php?id=1532}{Sensibilization
  to Research: MDE} (FIL-A3)
\item
  \href{https://campusneo.mines-nantes.fr/campus/course/view.php?id=621}{PRIME
  Project} (FING-A1)
\item
  \href{https://campusneo.mines-nantes.fr/campus/course/view.php?id=115}{PIST
  Project} (FING-A2)
\item
  \href{https://campusneo.mines-nantes.fr/campus/course/view.php?id=1540}{PFE
  Project} (FING-A3)
\item
  \href{https://campusneo.mines-nantes.fr/campus/course/view.php?id=1344}{PFE
  Project} (FIL-A3)
\end{itemize}

\hypertarget{section-8}{%
\section{2014-2015}\label{section-8}}

\begin{itemize}
\tightlist
\item
  \href{https://campusneo.mines-nantes.fr/campus/course/view.php?id=1472}{Domain-Specific
  Languages} (FING-A3-GSI, FIL-A3)
\item
  \href{https://campusneo.mines-nantes.fr/campus/course/view.php?id=1471}{Databases}
  (FING-A2)
\item
  \href{https://campusneo.mines-nantes.fr/campus/course/view.php?id=1532}{Sensibilization
  to Research: MDE} (FIL-A3)
\item
  \href{https://campusneo.mines-nantes.fr/campus/course/view.php?id=621}{PRIME
  Project} (FING-A1)
\item
  \href{https://campusneo.mines-nantes.fr/campus/course/view.php?id=391}{IPIPIP
  Project} (FING-A1)
\item
  \href{https://campusneo.mines-nantes.fr/campus/course/view.php?id=1486}{PFE
  Project (FING)} (FING-A3)
\item
  \href{https://campusneo.mines-nantes.fr/campus/course/view.php?id=1344}{PFE
  Project (FIL)} (FIL-A3)
\end{itemize}

\hypertarget{section-9}{%
\section{2013-2014}\label{section-9}}

\begin{itemize}
\tightlist
\item
  \href{https://campusneo.mines-nantes.fr/campus/course/view.php?id=1415}{Domain-Specific
  Languages} (FING-A3-GSI, FIL-A3)
\item
  \href{https://campusneo.mines-nantes.fr/campus/course/view.php?id=1327}{Databases}
  (FING-A2)
\item
  \href{https://campusneo.mines-nantes.fr/campus/course/view.php?id=1532}{Sensibilization
  to Research: MDE} (FIL-A3)
\item
  \href{https://campusneo.mines-nantes.fr/campus/course/view.php?id=621}{PRIME
  Project} (FING-A1)
\item
  \href{https://campusneo.mines-nantes.fr/campus/course/view.php?id=391}{IPIPIP
  Project} (FING-A1)
\item
  \href{https://campusneo.mines-nantes.fr/campus/course/view.php?id=115}{PIST
  Project} (FING-A2)
\item
  \href{https://campusneo.mines-nantes.fr/campus/course/view.php?id=1305}{PFE
  Project} (FING-A3)
\item
  \href{https://campusneo.mines-nantes.fr/campus/course/view.php?id=1344}{PFE
  Project} (FIL-A3)
\end{itemize}

\hypertarget{section-10}{%
\section{2012-2013}\label{section-10}}

\begin{itemize}
\tightlist
\item
  \href{https://campusneo.mines-nantes.fr/campus/course/view.php?id=1133}{Databases}
  (FING-A2)
\item
  \href{https://campusneo.mines-nantes.fr/campus/course/view.php?id=621}{PRIME
  Project} (FING-A1)
\item
  \href{https://campusneo.mines-nantes.fr/campus/course/view.php?id=391}{IPIPIP
  Project} (FING-A1)
\end{itemize}

\hypertarget{section-11}{%
\section{2011-2012}\label{section-11}}

\begin{itemize}
\tightlist
\item
  \href{https://campusneo.mines-nantes.fr/campus/course/view.php?id=391}{IPIPIP
  Project} (FING-A1)
\end{itemize}

\hypertarget{section-12}{%
\section{2010-2011}\label{section-12}}

\begin{itemize}
\tightlist
\item
  \href{http://web.emn.fr/x-info/atlanmod/index.php?title=The_MDE_Diploma}{Higher-order
  Transformations} (MDE Diploma)
\item
  \href{https://campusneo.mines-nantes.fr/campus/course/view.php?id=391}{IPIPIP
  Project} (FING-A1)
\end{itemize}

\hypertarget{and-before}{%
\section{2009 and before}\label{and-before}}

I have given lessons in the following courses at Politecnico di Milano.

\begin{itemize}
\tightlist
\item
  Operating Systems Project, Lecturer, 2006-2007, 2008-2009
\item
  Technological culture, Lecturer, 2007-2008
\item
  Software Engineering, Teaching Assistant, 2008-2009
\item
  Web Technologies, Teaching Assitant, 2007-2008, 2008-2009
\item
  Computer Science 3 (Algorithms and Data Structures), Teaching
  Assistant, 2006-2007, 2007-2008, 2008-2009
\item
  Web Technologies, Teaching Assistant for the post-university master on
  Service Oriented Architectures, 2006-2007, 2007-2008
\item
  Information Systems, Teaching Assistant for the on-line degree,
  2005-2006
\end{itemize}

}

\newpage
\section{Publication list}

\nocitejournal{*}
\bibliographystylejournal{unsrt}
\bibliographyjournal{MyJournals}

\nociteproceedings{*}
\bibliographystyleproceedings{unsrt}
\bibliographyproceedings{MyProceedings}

\nociteconference{*}
\bibliographystyleconference{unsrt}
\bibliographyconference{MyConferences}

\nociteworkshop{*}
\bibliographystyleworkshop{unsrt}
\bibliographyworkshop{MyWorkshops}

\nocitetechnical{*}
\bibliographystyletechnical{unsrt}
\bibliographytechnical{MyTechnicalReports}

\end{document}
