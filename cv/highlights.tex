
\subsection{Short bio}

Since 2010, I hold an Associate Professor (Maitre Assistant) position at IMT Atlantique (formerly Ecole des Mines de Nantes). 
I obtained my M.Sc. in Computer Science and Engineering at Politecnico di Milano in 2005 (100/100) and my Ph.D. in Information Engineering (mark A) in April 2009, with a thesis on Model Transformations for Artifact Generation in Model-Driven Environments (advisor prof. Piero Fraternali).

Since the beginning of my career, my research focused on the design, implementation and evaluation of novel methods and tools for modeling software systems and reasoning on software models. The scientific work, which resulted in high-profile publications, has always been supported by substantial implementation and demonstration efforts, and by a continuos interaction with industry. While during my doctoral studies I was focused on modeling a precise application domain (data-intensive Web applications), moving to IMT Atlantique I expanded the spectrum of modeled applications. Recently I have started to look outside of purely software systems, by considering hybrid hardware/software models for cyber-physical systems.

During my doctoral studies I have been visiting researcher at the McGill university (Montreal, Canada, 2007), where I came in contact with multi-paradigm heterogeneous modeling. Later I have been several times a visiting researcher at the National Institute of Informatics (NII) in Tokyo, Japan, where I collaborated on advancing the state of the art in bidirectional transformations.

\subsection{Publications and visibility}

I am widely recognized as an expert in model transformation, notably for extending the model-transformation paradigm to higher-order model transformations, and for my contributions to the ATL language since 2010. In particular, my article "On the Use of Higher-Order Model Transformations" has collected 252 citations since 2009 (up to June 2023 - Google Scholar). I have an h-index of 26 (June 2023 - Google Scholar).

\medskip
My top five publications are:
\begin{itemize}
\item Zheng Cheng, Massimo Tisi, Remi Douence. CoqTL: a Coq DSL for rule-based model transformation. Software and Systems Modeling 19 (2), 2020, pages 425-439, Springer.
\item Soichiro Hidaka, Frederic Jouault, Massimo Tisi: On Additivity in Transformation Languages. International Conference on Modeling Languages and Systems, MODELS 2017.
\item Amine Benelallam, Abel Gómez, Massimo Tisi, and Jordi Cabot. Distributing relational model transformation on MapReduce. Journal of Systems and Software, 142:1–20, 2018. Elsevier.
\item Soichiro Hidaka, Massimo Tisi, Jordi Cabot, Zhenjiang Hu: Feature-based classification of bidirectional transformation approaches. Software and System Modeling, Volume 15(3), 2016, Pages 907-928, Springer.
\item Massimo Tisi, Frédéric Jouault, Piero Fraternali, Stefano Ceri, Jean Bézivin: On the Use of Higher-Order Model Transformations. Proceedings of the 5th European Conference on Model Driven Architecture - Foundations and Applications, 2009, Pages 18–33.
\end{itemize}

\medskip
Awards: 
\begin{itemize}
\item Nominee for Best Paper Award in the European Joint Conferences on Theory and Practice of Software 2017 (ETAPS 2017) for the article: Zheng Cheng, Massimo Tisi, A Deductive Approach for Fault Localization in ATL Model Transformations.
\item Nominee for Best Paper Award in the International Conference for Software Testing 2017 (ICST 2017) for the article: Zheng Cheng, Massimo Tisi, Incremental Deductive Verification for Relational Model Transformations.
\item Best Performance Award for the ATL solution to the Transformation Tool Contest 2017 (TTC 2017), at the Software Technologies: Applications and Foundations Conference 2017 (STAF 2017).
\item Best paper award in the International Conference on Web Engineering 2009 (ICWE 2009) for the article: Fraternali, P., Tisi, M.: A higher-order generative framework for weaving traceability links into a code generator for Web application testing.
\item Best paper award in the International Workshop on Model Driven Web Engineering 2008 (MDWE 2008) for the article: Brambilla, M., Fraternali, P., Tisi, M.: A Metamodel Transformation Framework for the Migration of WebML Models to MDA.
\end{itemize}

\subsection{Funding and Project Management}
I have worked in several national (Italian and French), and European research projects, as well as with several companies for research exploitation and technology transfer (including Sodifrance, Atos, Thales, and several SMEs). I have been the principal investigator in AtlanMod for four projects, and I have secured 430K Euro in fundings from European and national projects.

\medskip
Highlights:
 \begin{itemize}
 \item Coordinator for the \href{https://www.lowcomote.eu/}{Lowcomote} project: Training the Next Generation of Experts in Scalable Low-Code Engineering Platforms (2019-2023: Marie Curie International Training Network).
 \item Work-package leader in the MONDO project: Scalable Modeling and Model Management on the Cloud (2013-2016: 7th Framework Programme, STREP 2012). Mention: \textbf{Excellent}.
 \item Principal investigator for Ecole des Mines de Nantes in the AutoMobile project: Automated Mobile App Development (2013-2015: 7th Framework Programme, Research for SMEs). Mention: \textbf{Excellent}.
\end{itemize}

\subsection{Research Community Service}

I have served as General Chair of the Software Technologies: Applications and Foundations 2022 (STAF 2022) Federated Conference. I co-organized succesful workshop series on Scalability in Model Driven Engineering (BigMDE) and Modeling in Low-Code Development Platforms (LowCode). 
I have served in the program committee of several international conferences and workshops in software engineering, including the leading conferences on modeling (MODELS), model transformation (ICMT) and software language engineering (SLE). I was Artifact Chair in MODELS, and co-organizer of the Project Showcase at STAF.

I reviewed for top journals in software engineering, including Transactions of Software Engineering, Journal of Software: Evolution and Process, Software and System Modeling, Science of Computer Programming. I was editor for a special section of the Software and System Modeling journal.

% Since 2013 I organize the BigMDE workshop on scalable model-driven engineering, and this year I host the Projects Showcase at the STAF federated conferences 2017.

I am member of the Monitoring and Evaluation Committee at the Pôle Images \& Réseaux.

\subsection{Teaching and Academic Supervision}

Since 2013 I am lecturer of Domain-Specific Languages and Model-Driven Engineering (master level) and of Databases (bachelor level) at IMT Atlantique. I co-advised 10 PhD students and 7 postdocs and research engineers.

I am member of the council of the SPIN doctoral school, and of co-responsible of the Doctoral Affairs at the LS2N laboratory.

\subsection{Main research areas}

\subsection*{Model Transformation Languages}
In the scope of model-driven engineering, model transformation aims to provide a mean to specify the way to produce target models from a number of source models. For this purpose, it enables developers to define the way source model elements must be matched and navigated in order to initialize the target model elements. Formally, a model transformation has to define the way for generating a model Mb, conforming to a metamodel MMb, from a model Ma conforming to a metamodel MMa. In practice, model transformations are widely used in model-driven engineering, e.g. during code generation (their originally intended use case), translation among modeling languages, data translation for interoperability, controlled updates over a model at runtime.

I have investigated different paradigms for model transformation, including the relational paradigm (exemplified by the ATL transformation language), the functional paradigm (exemplified by the OCL language), the data-flow paradigm (exemplified by the fUML language). I have studied correspondences between these paradigms and translations among them. 

I have contributed at clarifying the properties of transformation languages, especially bidirectionality (by feature modeling the design space of bidirectional transformations) and recently additivity, a property relating the syntactic constructs of the transformation specification to their semantic output.  
Finally I have distinguished the important class of higher-order transformations, i.e. transformations that manipulate other transformationss

\subsection*{Model Transformation Verification}

With the increasing use of model-driven engineering in safety-critical domains (e.g., in automotive industry, medical data processing, aviation), it is crucial to develop techniques and tools that prevent incorrect model transformations from generating faulty models. The effects of such faulty models could be unpredictably propagated into subsequent MDE steps, e.g. code generation.

For this reason, there is a great need of mechanisms to ensure quality and the absence of errors in models and model transformations. Verification is one of the effective techniques that are typically used to achieve this.

Verification of models and model transformations refers to the ability of these elements to satisfy one or more correctness properties. These properties express certain characteristics that the element under analysis must feature in order to be considered correct. It is typical for verification tools to use formal methods to determine whether the model or model transformation under analysis satisfies the correctness properties under scrutiny.

I have contributed to formalize the execution semantics of the ATL language by translation to the Boogie intermediate verification language. Based on this semantics I have designed a fault localization method, capable of pinpointing the faulty lines of a transformation that does not respect a contracts.

I have designed a new technique for the decomposition of transformation contracts, to improve the efficiency of the automatic verification process on a SMT solver. I have contributed an incremental verification technique, so that after a change to the system, only the impacted part needs to be reverified. This idea greatly increases the applicability of transformation verification in industrial scenarios. 

\subsection*{Scalability of Model-Driven Engineering}
The increasing adoption of Model-Driven Engineering in industrial contexts highlights scalability as a critical limitation. Indeed, several Model-driven tools show critical efficiency limitations in handling very large models (VLMs), e.g. models made by millions of model elements, not unusual in real-life industrial scenarios. Examples of such models appear both at development time, e.g. while reverse-engineering big systems and at runtime, e.g. coming from a set of sensors, from OpenData repositories or when building applications on social networks. Moreover, the proliferation of models produced as input-outputs of software engineering tasks at development/maintenance time also highlights scalability problems in the management of the model artifacts.

In order to tackle the scalability problem I research solutions on three main axes:

\begin{itemize}

\item Efficient Transformation. Performing only the strictly required computation on models improves scalability, as only parts of VLM need to be loaded and manipulated. I provided the following support for the efficient execution of model-to-model transformations:
1) An engine for the incremental execution of model-to-model transformations, performing only the necessary re-computations; 2) A lazy execution semantics for model transformations, delaying the computation to when it is strictly needed; 3) An engine for the event-driven reactive execution of model transformations, where the engine performs only the computation needed to react to model updates and requests from the user application.

\item Parallel Transformation. The previous techniques increase efficiency by avoiding unnecessary computation. However, this is not always possible, as several transformations perform global algorithms on the whole model. In this scenario, a solution for the scalability problem would be the parallelization of the transformation with the aim of  decomposing it in smaller independent problems, more easily manageable with current tools. In this context, I conceived 1) A shared-memory implicit parallel execution engine for the ATL transformation language and 2) A distributed framework for model transformation on the Map-Reduce programming model. In particular automatic model distribution is performed within the cluster, by a streaming partitioning algorithm over the input model.

\item Efficient Model Persistence. Very large models need high-performance mechanism for their storage and access. Traditional approaches are file-based, especially XML-based, or rely on relational databases. I investigated innovative mechanisms for model storage, by relying on map-databases, graph-databases and distributed hash-tables. These approches have recently converged in the NeoEMF model persistence tool.

\end{itemize}
