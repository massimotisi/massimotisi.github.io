
\subsection*{Short bio}

Since 2010, I hold an Associate Professor position in the Department of Automation, Production and Computer Science at IMT Atlantique (formerly Ecole des Mines de Nantes). I’m team leader of the Naomod team (LS2N, UMR CNRS 6004). I obtained the Habilitation to Direct Research (HDR) in December 2023.

I obtained my M.Sc. in Computer Science and Engineering at Politecnico di Milano in 2005 (100/100) and my Ph.D. in Information Engineering (grade A) in April 2009, with a thesis on Model Transformations for Artifact Generation in Model-Driven Environments (advisor prof. Piero Fraternali).

Since the beginning of my career, my research focused on the design, implementation and evaluation of novel methods and tools for modeling software systems and reasoning on software models. The scientific work, which resulted in high-profile publications, has always been supported by substantial implementation and demonstration efforts, and by a continuos interaction with industry. While during my doctoral studies I was focused on modeling a precise application domain (data-intensive Web applications), moving to IMT Atlantique I expanded the spectrum of modeled applications, and recently included cyber-physical systems and digital twins. 

Recently I worked at advancing the state of the art on low-code platforms, i.e. development platforms in the cloud that use software modeling to enable application development by end users and domain experts. I contribute at improving their scalability and integration with information systems, AI components, and digital twins. 

During my doctoral studies I have been visiting researcher at the McGill university (Montreal, Canada, 2007), where I came in contact with multi-paradigm heterogeneous modeling. Later I have been several times a visiting researcher at the National Institute of Informatics (NII) in Tokyo, Japan, where I collaborated on advancing the state of the art in bidirectional transformations.

\subsection*{Publications and visibility}

I am widely recognized as an expert in model transformation, notably for introducing higher-order model transformations, and for my contributions to the ATL language since 2010. In particular, my article "On the Use of Higher-Order Model Transformations" has collected 252 citations since 2009. I have been cited more than 2300 times, and have an h-index of 26 (June 2023 - Google Scholar). I am also recognized as a leading researcher on low-code platforms, since I was among the first academics to work on low-code, by coordinating an European project, organizing an ongoing workshop series, and publishing top level publications on the subject. 

\medskip
My top five publications are:
\begin{itemize}
\item Davide Di Ruscio, Dimitirs Kolovos, Juan de Lara, Alfonso Pierantonio, Massimo Tisi, Manuel Wimmer. Low-code development and model-driven engineering: Two sides of the same coin? Software and Systems Modeling Journal 21 (2), 2022, pages 437-446, Springer. \textbf{97 citations in the last year}
\item A Model-Driven Methodology to Accelerate Software Engineering in the Internet of Things. Imad Berrouyne, Mehdi Adda, Jean-Marie Mottu, Massimo Tisi. IEEE Internet of Things Journal 9 (20), 2022, 19757-19772, IEEE. \textbf{10.2 impact factor}
\item Zheng Cheng, Massimo Tisi, Joachim Hotonnier: Certifying a rule-based model transformation engine for proof preservation. Proceedings of the International Conference on Modeling Languages and Systems, MODELS 2020, Springer. \textbf{rank A}
\item Soichiro Hidaka, Massimo Tisi, Jordi Cabot, Zhenjiang Hu: Feature-based classification of bidirectional transformation approaches. Software and System Modeling Journal, Volume 15(3), 2016, Pages 907-928, Springer. \textbf{106 citations} 
\item Massimo Tisi, Frédéric Jouault, Piero Fraternali, Stefano Ceri, Jean Bézivin: On the Use of Higher-Order Model Transformations. Proceedings of the 5th European Conference on Model Driven Architecture - Foundations and Applications, 2009, Pages 18–33, Springer. \textbf{264 citations}
\end{itemize}

\medskip
My top five awards are: 
\begin{itemize}
\item Nominee for Best Paper Award in the European Conference on Modelling Foundations and Applications (ECMFA 2023) for the article: Josselin Enet, Erwan Bousse, Massimo Tisi, Gerson Sunyé. Protocol-Based Interactive Debugging for Domain-Specific Languages.
\item Nominee for Best Paper Award in the European Joint Conferences on Theory and Practice of Software 2017 (ETAPS 2017) for: Zheng Cheng, Massimo Tisi, A Deductive Approach for Fault Localization in ATL.
\item Nominee for Best Paper Award in the International Conference for Software Testing 2017 (ICST 2017) for: Zheng Cheng, Massimo Tisi, Incremental Deductive Verification for Relational Model Transformations.
%\item Conciseness Award for the ATL-OCL solution to the Transformation Tool Contest 2018 (TTC 2018), at the Software Technologies: Applications and Foundations Conference 2018 (STAF 2018).
\item Best Performance Award for the ATL solution to the Transformation Tool Contest 2017 (TTC 2017), at the Software Technologies: Applications and Foundations Conference 2017 (STAF 2017).
\item Best paper award in the International Conference on Web Engineering 2009 (ICWE 2009) for the article: Piero Fraternali, Massimo Tisi: A higher-order generative framework for weaving traceability links into a code generator for Web application testing.
%\item Best paper award in the International Workshop on Model Driven Web Engineering 2008 (MDWE 2008) for the article: Brambilla, M., Fraternali, P., Tisi, M.: A Metamodel Transformation Framework for the Migration of WebML Models to MDA.
\end{itemize}

\subsection*{Funding and Project Management}
I have worked in several national (Italian and French), and European research projects, as well as with several companies for research exploitation and technology transfer (including Sodifrance, Atos, Thales, and several SMEs). I have secured more than 1.5M Euro in fundings from European and national projects.
% 747 (Lowcomote) + 342 (Mondo)

\medskip
Highlights:
 \begin{itemize}
 \item Coordinator for the \href{https://www.lowcomote.eu/}{Lowcomote} project: Training the Next Generation of Experts in Scalable Low-Code Engineering Platforms (2019-2023: Marie Curie International Training Network).
 \item Work-package leader in the MONDO project: Scalable Modeling and Model Management on the Cloud (2013-2016: 7th Framework Programme, STREP 2012). Mention: \textbf{Excellent}.
 \item Principal investigator for Ecole des Mines de Nantes in the AutoMobile project: Automated Mobile App Development (2013-2015: 7th Framework Programme, Research for SMEs). Mention: \textbf{Excellent}.
\end{itemize}

\subsection*{Research Community Service}

I have served as General Chair of the Software Technologies: Applications and Foundations 2022 (STAF 2022) Federated Conference. I co-organized succesful workshop series on Scalability in Model Driven Engineering (BigMDE) and Modeling in Low-Code Development Platforms (LowCode). 
I have served in the program committee of several international conferences and workshops in software engineering, including the leading conferences on modeling (MODELS), model transformation (ICMT) and software language engineering (SLE). I was Artifact Chair in MODELS, and co-organizer of the Project Showcase at STAF.

I am a member of the editorial board of the Software and System Modeling journal, leading journal in the modeling community. I reviewed for top journals in software engineering, including Transactions of Software Engineering, Journal of Software: Evolution and Process, Software and System Modeling, Science of Computer Programming. I was editor for a special section of the Software and System Modeling journal.

I am member of the Monitoring and Evaluation Committee at the Pôle Images \& Réseaux.

\subsection*{Teaching and Academic Supervision}

I am co-responsible for the Apprenticeship Engineer, Software Engineering Specialisation at IMT Atlantique. Since 2013, I am responsible for several courses at IMT Atlantique, including Domain-Specific Languages, Model-Driven Engineering, Language Engineering and Databases.

I co-advised 9 PhD students (1 \textbf{excellent} mention from the partner Canadian university) and 7 postdocs and research engineers. I am member of the council of the SPIN doctoral school, and co-responsible of the Doctoral Affairs at the LS2N laboratory.
